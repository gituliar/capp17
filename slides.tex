\documentclass[11pt]{article}

\usepackage[%
  papersize={160mm,120mm},
  bottom=2mm,
  top=2mm,
  textwidth=150mm,
  textheight=115mm,
  hcentering,
]{geometry}

% Mathtools with AMS Math
\usepackage{mathtools}

% Use TeX Gyre Schola font family with Fourier math fonts
\usepackage[T1]{fontenc}
\usepackage{tgschola}
\usepackage{fouriernc}
\usepackage{parskip}
\setlength{\parskip}{0.25em}
\renewcommand{\baselinestretch}{0.9}

\usepackage{enumitem}
\setlist{itemsep=.05em,parsep=.25em}

\usepackage{graphicx}
\usepackage{amsmath, caption, fancybox, soul,xspace}
\usepackage[dvipsnames]{xcolor}
\usepackage{hyperref}
\hypersetup{colorlinks,linkcolor=,urlcolor=blue}

\usepackage{axodraw2}

\usepackage{fancyvrb}

\usepackage{minibox}

\newcommand{\be}{\begin{equation}}
\newcommand{\bi}{\begin{itemize}}
\newcommand{\bn}{\begin{enumerate}}
\newcommand{\bv}{\begin{Verbatim}}
\newcommand{\ee}{\end{equation}}
\newcommand{\ei}{\end{itemize}}
\newcommand{\en}{\end{enumerate}}
\newcommand{\ev}{\end{Verbatim}}

\newcommand{\fire}{\texttt{FIRE}\xspace}
\newcommand{\fuchsia}{\texttt{Fuchsia}\xspace}
\newcommand{\hyperint}{\texttt{HyperInt}\xspace}
\newcommand{\litered}{\texttt{LiteRed}\xspace}
\newcommand{\reduze}{\texttt{Reduze2}\xspace}

\newcommand{\blue}[1]{{\color{blue} #1}}
\newcommand{\green}[1]{{\color{ForestGreen} #1}}

\newcommand{\x}{\;}
\newcommand{\vs}{\vspace{2mm}}

\newcommand{\abs}[1]{\left|#1\right|}
\newcommand{\D}{\mathrm{d}}
\newcommand{\eps}{\epsilon}
\newcommand{\linux}{\texttt{Linux}\xspace}
\newcommand{\maple}{\texttt{Maple}\xspace}
\newcommand{\maxima}{\texttt{Maxima}\xspace}
\newcommand{\maximasage}{\texttt{Maxima/Sage}\xspace}
\newcommand{\python}{\texttt{Python}\xspace}
\newcommand{\rank}{\mathrm{rank}}
\newcommand{\sage}{\texttt{SageMath}\xspace}
\newcommand{\code}[1]{\texttt{#1}}
\newcommand{\F}[1]{\texttt{#1}} % use this to style function names
\newcommand{\V}[1]{\mathbf{#1}} % use this to style vector names
\newcommand{\prompt}[2]{\textcolor{prompt}{#1} \textcolor{command}{#2}}
\newcommand{\functionitem}[2]{\item[$\F{#1}(#2)$\hfill\textit{(function)}]}
\newcommand{\classitem}[1]{\item[$#1$\hfill\textit{(class)}]}


\newcommand{\refx}[1]{\href{1}{\scriptsize #1}}
\newcommand{\titlebox}[1]{\centerline{\shadowbox{\begin{Bcenter}{\Large #1}\end{Bcenter}}}}
\newcommand{\titleboxx}[2]{\centerline{\shadowbox{\begin{Bcenter}\Large #1\vspace{1mm}\\ {\color{white}q}#2{\color{white}q}\end{Bcenter}}}}

%\setmonofont{Go-Mono}
\definecolor{textcolor}{RGB}{60,60,60}

\begin{document}

\color{textcolor}

\begin{center}
  \mbox{}
  \vfill
  {\LARGE \bf \scshape Introduction \\ \Large to \LARGE \\ Differential Equations \\ \Large for \LARGE \\ Feynman Integrals \\}
  \vfill
  {\large Oleksandr Gituliar} \\ \href{mailto:oleksandr@gituliar.net}{oleksandr@gituliar.net}
  \vfill
  \includegraphics[height=1.2cm]{img/logo_uhh}\\
  II. Institut f\"ur Theoretische Physik \\ Universit\"at Hamburg
  \vfill
  {Computer Algebra in Particle Physics 2017 \\ DESY (Hamburg)}
  \vfill
  \mbox{}
\end{center}
\newpage
 

\titlebox{Plan for Today}
\begin{enumerate}
  \item Methods for caluclating Feynman integrals (20')
  \begin{itemize}
    \item Integration by Parts (IBP)
    \item Method of Differential Equations (MDE)
  \end{itemize}
  \item MDE \& Solution Techniques (30')
  \begin{itemize}
    \item ordinary vs partial
    \item epsilon form
    \item boundary conditions
  \end{itemize}
  \item Examples (30')
  \item MDE \& Limitations (10')
\end{enumerate}
\newpage

\titlebox{I.1 Feynman Integrals Calculus}
\bi \item Feynman/Schwinger parameters
          \bi \item \hyperint and hyperlogarithms
              \ei
    \item Mellin-Barnes (MB) representation
          \bi \item popular for massive diagrams
              \ei
    \item Integration-by-parts (IBP) reduction
          \bi \item universal and easy-to-use
              \item \fire, \litered, \reduze
              \ei
    \item Method of differential equations (MDE)
          \bi \item huge potential
              \ei
    \ei
\newpage

\titlebox{I.2 Method of Differential Equations}
\bn
  \item Find IBP rules
    \bi
      \item highly automated
      \item \fire, \litered, \reduze
    \ei
  \item Derive ODE
  \item Find $\eps$-form
    \bi
      \item already very powerful (less-understood than IBP)
      \item Lee method: \fuchsia (2016), \texttt{epsilon} (2017)
      \item other:
    \ei
  \item Solve ODE
  \item Find constants of integration
    \bi
      \item depends on the problem
      \item may be the most complex task
    \ei
\en
\newpage

\titleboxx{Example I}{One-Loop Massive Self-Energy}
%{\small
%{\bf Scalar integrals} are the most {\em convenient} to compute.
%Therefore, we assume that integrals with vector or spinor indices (e.g. amplitudes) are already expressed as a combination of scalar integrals and constant objects which reveal the index structure.}
\bi
  \item {\bf Functions}: {\em from vectors to scalars}
$$ \raisebox{-28pt}{
     \SetScale{0.5}
     \begin{axopicture}(200,120)
       \SetOffset(0,10)
       \DoubleArc[arrow](100,50)(40,0,180){2}
       \Text(100,105){$l$}
       \DoubleArc[arrow](100,50)(40,180,360){2}
       \Text(100,-5){$l-p$}
       \Text(30,70){$p$}
       \Gluon(0,50)(60,50){5}{4}
       \Vertex(60,50){2}
       \Text(170,70){$p$}
       \Gluon(140,50)(200,50){5}{4}
       \Vertex(140,50){2}
     \end{axopicture}}
   =
 \Pi^{\mu\nu}_{ab}({p^2},m)
   =
   \delta_{ab} \x \left(p^\mu p^\nu - g^{\mu\nu} p^2\right) \; \Pi({p^2},m)
$$
\vspace{-3mm}
$$
   \Pi(\green{p^2},m)
   =
   \int \D^n l \; F(p,l,m)
$$
  \item {\bf Arguments}: {\em from vectors to scalars}
        $$ F(p,l,m) \;\to\; F(\blue{l^2},\; \blue{l \!\cdot\! p}, \; \green{p^2}, \; m) $$
  \item In general, the number of \blue{scalar integration variables} is given by
        $$N(L,E) = \frac{L\,(L+1)}{2} + L\,E \quad \sim \quad \mathcal{O}(L^2) \;\;\leftarrow\;\; \parbox{50mm}{another source of growing complexity at higher orders}$$
        where $E$ -- number of {\em external momenta}, $L$ -- number of {\em loop momenta}
  \bi
    \item 1-loop propagator: $\blue{N(1,1)=2}$
    \item 4-loop propagator: $N(4,1)=14$ (ask Jos Vermaseren about details)
  \ei
\ei
\newpage


\bi
  \item The problem contains two denominators 
        $$\quad D_1 = l^2-m^2 \quad \quad D_2 = (l-p)^2-m^2  $$
        which map into our integration invariants in a unique way 
        $$ F(p,l,m) \;\to\; F(\blue{l^2},\; \blue{l \!\cdot\! p}, \; p^2, \; m) \to F(\blue{D_1}, \blue{D_2}, p^2,m) $$
  \item One integral family\footnote{since now we do not write $p^2$ and $m$ explicitly}
        $$F(n_1, n_2) = \int \D^n l \frac{1}{D_1^{n_1} D_2^{n_2}}$$
\ei

\vspace{4mm}
\begin{Verbatim}[fontsize=\small,frame=lines, label=\fbox{Mathematica}]

  <<LiteRed`

  SetDim[n];
  Declare[{m2}, Number, {l,p}, Vector];
  NewBasis[$b, {sp[l]-m2, sp[l-p]-m2}, {l}];

\end{Verbatim}
\newpage


\titleboxx{Example I}{Master Integrals}
\begin{Verbatim}[fontsize=\small,frame=lines, label=\fbox{Mathematica}]

  AnalyzeSectors[$b];
  FindSymmetries[$b];
  SolvejSectors /@ UniqueSectors[$b]

  MIs[$b]

  > {j[$b,0,1], j[$b,1,1]}
\end{Verbatim}
\bi
  \item We obtain two master integrals
        $$F_1 = F(0,1) \qquad F_2 = F(1,1)$$
  \item Any other integral is a linear combination of only these two, e.g.,
        $$F(2,1) = \frac{n-2}{2 m^2 (p^2-4m^2)} F_1 + \frac{n-3}{p^2-4m^2} F_2$$
\ei
\newpage


\titleboxx{Example I}{Differential Equations}

\vs
\begin{Verbatim}[fontsize=\small,frame=lines, label=\fbox{Mathematica}]

$ds = Dinv[#,sp[p,p]]& /@ MIs[$b] // IBPReduce;
$ode = Coefficient[#, MIs[$b]]& /@ $ds;
\end{Verbatim}
\vs

\bi
  \item This code produces a system of differential equations
  $$
    \begin{aligned}
     \frac{\D F_1}{\D p^2} & = 0
     \\
     \frac{\D F_2}{\D p^2} & = \frac{2-2\eps}{s\x(s-4m^2)} F_1 + \frac{2m^2-\eps s}{s\x(s-4m^2)} F_2
    \end{aligned}
  $$
  where $s=p^2$ and $n=4-2\eps$
\ei
\vs

This system is simple and we could solve it right away using {\em <your favourite>} method.
Today, I want to demonstrate you how this and many other systems can be solved throug using their {\bf $\boldsymbol{\eps}$-form}.
As you will see this is a highly automated task.

\vs
\fbox{Exercise 1a.} \\
Derive another system of differential equations, but this time in~$m^2$. (Hint: instead of \code{Dinv} use \code{Fromj}, \code{D}, and \code{Toj} functions ).
\newpage


\titleboxx{Example I}{$\eps$-form}
\newpage


\titlebox{Holonomic Functions}
A function $f=f(x)$ is called {\em holonomic} if there exist polynomials $a_n(x),\x\ldots,\x a_0(x)$ such that
\be
  \label{eq:ode}
  a_n(x) f^{(n)} - a_{n-1}(x)\x f^{(n-1)} - \ldots - a_0(x) \x f = 0
\ee
holds for all $x$. Hence, the holonomic function is uniquely defined by
\bi
  \item the differential equation
  \item a number of initial values $f(x_0),\x f'(x_0),\x \ldots,\x f^{(n-1)}(x_0)$
\ei

\vs
\underline{Examples} of holonomic functions:
\bi
  \item {\em all algebraic functions}
  \item \fbox{\em Generalized Hypergeometric functions}
    \bi
      \item {\em polylogarythms}
      \item {\em Elliptic functions}
    \ei
  \item {\em Bessel functions}
  \item {\em Airy functions}
  \item {\em Legendre and Chebyshev polynomials}
  \item {\em Heun functions}
  \item and many others that have no name and no closed form
\ei

\vs\vs
For more details, see the lecture by M.Kauers at CAPP 2011.
\newpage

\titlebox{Holonomic Functions}
A function $f=f(x)$ is called {\em holonomic} if there exist polynomials $a_n(x),\x\ldots,\x a_0(x)$ such that
$$
  a_n(x) f^{(n)} - a_{n-1}(x)\x f^{(n-1)} - \ldots - a_0(x) \x f = 0
$$
holds for all $x$. Hence, the holonomic function is uniquely defined by
\bi
  \item the differential equation
  \item a number of initial values $f(x_0),\x f'(x_0),\x \ldots,\x f^{(n-1)}(x_0)$
\ei

\vspace{1cm}
\fbox{Conclusion}
\vs
\bn
\item With the help of polynomials and differential equations we can define a large set of complicated functions which have no closed form and usually have a non-trivial integration representation.
\item These functions appear in the Feynman integrals.
\item In some situation it is much easier to solve ODE (than to directly integrate). \\
      In the rest of my talk I will show you how to do that.
\en
\newpage

\titlebox{Fuchsian Differential Equations}
\newpage


\titlebox{$n^{\text{th}}$-order ODE vs $n\times n$ system}

We can easily rewrite a $n^{\text{th}}$-order linear ODE given by
\be
  \label{eq:ode}
  y^{(n)} - a_1(x)\x y^{(n-1)} - \ldots - a_n(x) \x y = 0
\ee
as an $n\times n$ system of the form 
\be \frac{\D \bar{y}}{\D x} = A(x) \x \bar{y} \ee
where
\be
A(x) =
\begin{bmatrix}
  0 & 1 & \cdots & 0 & 0
  \\
  0 & 0 & \cdots & 0 & 0
  \\
  \vdots & \vdots & \ddots & \vdots & \vdots
  \\
  0 & 0 & \cdots & 0 & 1
  \\
  a_n(x) & a_{n-1}(x) & \cdots & a_2(x) & a_1(x)
\end{bmatrix}
\quad \text{and} \quad
\bar{y} =
\begin{pmatrix}
  y \\ y' \\ \vdots \\ y^{(n-2)} \\ y^{(n-1)}
 \end{pmatrix}
\ee
However, the inverse opperation is not as easy anymore.
\newpage


\titleboxx{Example II}{Expansion of Hypergeometric Functions}
The Generalized Hypergeometric Function
\be
 _{p+1}F_p
\left(
\begin{matrix}
  a_1,\x a_2,\x \ldots,\x a_{p+1}
  \\
  b_1,\x b_2,\x \ldots,\x b_{p}
\end{matrix}
;\x x
\right)
=
\prod_{i=1}^{p}\frac{\Gamma(b_i)}{\Gamma(a_i)\Gamma(b_i-a_i)}
\int_0^1 \frac{t_i^{a_i-1} (1-t_i)^{b_i-a_i-1}}{(1-x\x t_1\ldots t_{p})^{a_{p+1}}} \D t_i
\ee
is a solution to the differential equation
\be
\left[
 D \x (D+b_1-1)\x\cdots\x(D+b_p-1)\x - \x x\x(D+a_1)\x\cdots\x(D+a_{p+1})
\right] \x y = 0
\ee
where
$$D = x\frac{\D}{\D x}$$

\vs
\fbox{Exercise 2.}
Using your favourite CAS write a routine which for a given Generalized Hypergeometric Function, defined by the list $\{a_1,\ldots,a_{p+1},b_1,\ldots,b_p\}$, it returns a corresponding ODE, defined by the list $\{a_1(x),\ldots,a_p(x)\}$, in accordance with notation of eq.~\eqref{eq:ode}.

\newpage


\titleboxx{Example III}{NLO real correction to $2\to 2$ in QCD}
\newpage


\titlebox{A Word of Advice}
\bi
  \item Before computing, first think of how to check your results
  \item Read arXiv.org daily
  \bi
    \item subscribe to the mailing list
    \item e-mail To: physics@arxiv.org, Subject: subscribe
  \ei
  \item Search with scholar.google.com
  \item Do not be afraid to write e-mails
  \item Discuss your ideas
\ei
\newpage

\titlebox{Reading List}
\bi \item {\em "Feynman Integral Calculus"} by V.~Smirnov
    \item {\em "Lectures on Differential Equations for Feynman Integrals"} by J.~Henn
    \item {\em "Formal Power Series and Linear Systems of Meromorphic Ordinary Differential Equations"} by W.~Balser
    \item {\em "Computer Algebra in Particle Physics"} by S.~Weinzierl
    \item {\em "Introduction to Loop Calculations"} by G.~Heinrich
    \item {\em "Structure and Interpretation of Computer Programs"}\\by H.~Abelson and G.~Sussman with J.~Sussman
    \ei
\newpage

\titlebox{Summary}

\end{document}
